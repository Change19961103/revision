\documentclass{article}
\usepackage{parskip}
\usepackage[margin=2cm]{geometry}

% Thanks to Angus Pearson for nagging me to fuck about with fonts in
% XeLaTeX

\usepackage{fontspec} \usepackage{xltxtra}

% PT Serif
\setromanfont[ BoldFont=PTF75F.ttf, ItalicFont=PTZ56F.ttf,
BoldItalicFont=PTF76F.ttf, ]{PTF55F.ttf}

% Noto Sans
\setsansfont[ BoldFont=NotoSans-Bold.ttf,
ItalicFont=NotoSans-Italic.ttf, BoldItalicFont=NotoSans-BoldItalic.ttf
]{NotoSans-Regular.ttf}

\begin{document}

\textbf{\Huge Operating Systems Notes}

\subsection{Operating System Structure}




\section{Memory Managment}
Programs must be brought from disk into mmemory and then placed into a process.

The CPU can only access data from memory, not disk.

Register access takes at most 1 CPU clock, but Main Memory can take mayn cycles, causing a \emph{Stall}.

\subsection{Base and Limit Registers}
A set of \emph{base} and \emph{limit registers} define the logical adress space. the CPU must chech every
memory access is valid between the base and the limit for that user. Failure causes a trap to the OS monitor

\subsection{Virtual Address Space}
Logical/Virtual addresses are independent of physical memory.

Hardware translates virtual addresses into physical ones.

Logical/Virtual addresses a process can reference is called the address space.

\subsection{Memory Managment Unit (MMU)}
Effectively is a hash function from logical address to physical address.

a MMU prevents the need for swapped out process to be swapped back into the same physical addresses.

Swapping is not typically supported on mobile devices, more likely to to overwite least used data.


\subsection{Partitioning}
Main memory is usually broken up into two partitions; The OS and user process.

Each process is contained within a single contiguous section on memory.

Realocation registers are used to protect users processes from one another and from canging the OS code.

Some old techniques include:
\begin{itemize}
    \item Fixed Partitions - simple but causes fragmentation often
    \item Variable Partitions - no internal fragmentation, but can leave holes in the physical memory
\end{itemization}

Dynamic Storage-Allocation is possible using First-fit, Best-fir and Worst-fit in terms of hole filling.

\end{document}

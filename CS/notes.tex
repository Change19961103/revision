\documentclass{article}
\usepackage{parskip}
\usepackage[margin=2cm]{geometry}
\usepackage{hyperref}
\usepackage{fontspec}

% Thanks to Angus Pearson for nagging me to fuck about fonts in XeLaTeX

\usepackage{xltxtra}

% PT Serif
\setromanfont[
BoldFont=PTF75F.ttf,
ItalicFont=PTZ56F.ttf,
BoldItalicFont=PTF76F.ttf,
]{PTF55F.ttf}

% Noto Sans
\setsansfont[
BoldFont=NotoSans-Bold.ttf,
ItalicFont=NotoSans-Italic.ttf,
BoldItalicFont=NotoSans-BoldItalic.ttf
]{NotoSans-Regular.ttf}

\hypersetup{
  colorlinks = false
}

\begin{document}

\textbf{\Huge Computer Security Notes}\\
\textit{\footnotesize Found at: \href{http://benjaminshaw.uk}{benjaminshaw.uk}}

\section{Basics}

\subsection{A Definition of Security}

\textbf{Confidentiality}

Ensure that assets are accessed only by authorised parties.

\textbf{Integrity}

Assets can only be modified by authorised parties, or by authorised
means.

\textbf{Availability}

Assets are only accessible to authorised parties at the appropriate
times.

\textbf{Accountability}

Actions are traceable to those responsible

\textbf{Authentication}

User/data origin accurately identifiable

\subsection{Security Countermeasures}
  
\textbf{Prevention}

Stop security breaches via system design and defences

\textbf{Detection}

If a breach \underline{does} occur, detect it.

\textbf{Response}

A plan utilised when a breach is detected.

\subsection{Denial of Availability}

A user will expect that services be available to them. A common attack
is denying users this privilege. Denial of Service (DOS) attacks or
malware are two common ways of attacking availability.

\section{Cyber Security Essentials}
\filbreak
\subsection{Secure Configuration}

\textbf{Principles:}

Devices on a network should be configured such that they minimise the
number of inherent vulnerabilities.

Default settings can \textit{often} be insecure, which includes
default passwords.

\textbf{Actions:}

\begin{itemize}
\item
  Remove unnecessary user accounts, such as the \textit{Admin}
  account found on Windows XP installs.
\item
  Changing the default password
\item 
  Removal of unnecessary software
\item 
  Firewall software should regulate the incoming/outgoing connections on a device
\end{itemize}

\subsection{Boundary Firewalls \& Internet Gateways}

\textbf{Principles:}

Devices should be protected against unauthorised access and disclosure.

Firewalls are the first line of defence and can stop attacks before they even reach the network.

\textbf{Actions:}

\begin{itemize}
\item 
  Change default passwords
\item 
  Rules should be scrutinised before they are applied
\item 
  Unapproved services should be blocked by a rule
\item 
  Obsolete rules should be purged
\item 
  Firewall administration tools should not be accessible from outwith the network
\end{itemize}

\subsection{Access Control and Privilege Management}

\textbf{Principles:}

User accounts should have the minimum amount of privileges, with extended privileges awarded upon authorisation.

A compromised account with high levels of access can lead to a lot of damage.

\textbf{Actions:}

\begin{itemize}
\item 
  Account creation should be subjected to an approval process
\item 
  Administration accounts should only be used for legitimate administration purposes and \underline{not} activities that can be achieved with a standard account
\item 
  Elevated privilege accounts should require password changes periodically
\item 
  Users should be authenticated before being granted access to devices and applications
\item 
  Elevated accounts should be used when no longer required
\end{itemize}

\subsection{Patch Management}

\textbf{Principles:}

Remove unnecessary vulnerabilities by keeping software up-to-date.

\subsection{Malware Protection}


\end{document}
